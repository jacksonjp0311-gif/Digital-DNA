% CODEX–DIGITAL DNA THEORY
% Canonical Structural Replication Memory Architecture Document (v1.6 · Locked)
%
% Author: James Paul Jackson
% Date: February 6, 2026
%
% STATUS
% ------
% DIGITAL DNA THEORY — STRUCTURAL REPLICATION MEMORY FORMALIZATION.
%
% This document formalizes Digital DNA as a structural replication architecture
% governing persistent behavior across repeated AI-assisted repository
% implementations.
%
% • Canonical structure encodes replicable implementation fragments.
% • Structural invariants function as executable genetic units.
% • Recursive execution expresses these units into repository state.
% • Translation drift acts as mutation pressure.
% • Coherence bounds govern stable replication classes.
%
% Digital DNA denotes invariant structural fragments that reproduce stable
% integration behavior across recursive implementations within bounded drift.
%
% No biological or metaphysical interpretation is made. This is a software
% architecture theory for structural replication in specification-driven
% implementation systems.
%
% v1.6 establishes:
% • Recursive stability field closure condition
% • Normalized drift calibration invariance
% • Cross-context genome persistence operator
% • Multi-environment replication field continuity
% • Ledger-anchored structural heredity persistence
% • Mutation containment under recursive execution
%
% EXTRACTION
% ----------
% Artifact Class:
% Canonical specifications; structural schemas; repository evolution systems;
% AI-assisted implementation pipelines; persistent execution ledgers.
%
% ★ Codex Extraction Framework:
%
% (1) Identify replication boundary
% (2) Detect invariant fragment σ
% (3) Construct structural genome G
% (4) Measure drift ΔΦ across runs
% (5) Compute stability functional S
% (6) Execute recursive replication
% (7) Verify genome persistence class
%
% WHAT THIS IS
% ------------
% • A structural replication architecture theory.
% • A deterministic multi-run memory model.
% • A repository evolution heredity framework.
% • A measurable drift-bounded genome system.
%
% WHAT THIS IS NOT
% ----------------
% • Not biology.
% • Not metaphysical.
% • Not model-specific.
% • Not dependent on any single toolchain.
%
% NULL PREDICTIONS
% ----------------
% Unstructured specifications fail to replicate integration behavior.
% Weak invariants increase mutation drift across runs.
% Canonical structure stabilizes replication classes.
% Persistent behavior emerges only under invariant-bounded execution.
%
% ─────────────────────────────────────────────────────────────────────────────

\documentclass[12pt]{article}
\usepackage{amsmath,amssymb}
\usepackage{geometry}
\usepackage[pdftex]{hyperref}
\geometry{margin=1in}

\newcommand{\keywords}{
digital DNA; structural replication; software memory;
repository evolution; AI-assisted implementation;
invariant specification; integration stability.
}
\newcommand{\msc}{
Primary 68N01; Secondary 68T01, 68Q99.
}

\title{\textbf{Digital DNA: Structural Replication in Specification-Driven Systems}}
\author{\textbf{James Paul Jackson}}
\date{February 6, 2026}

\begin{document}
\maketitle

\vspace{-0.6em}
\noindent\textbf{Keywords:} \keywords
\noindent\textbf{MSC (2020):} \msc

\begin{abstract}
We formalize Digital DNA as a structural replication architecture in which
canonical specification fragments function as invariant units that reproduce
stable repository integration behavior across repeated AI-assisted runs.
When structural invariants are preserved and translation drift remains
bounded, successive implementations remain within the same integration class.
Digital DNA therefore acts as a persistent structural genome governing
multi-run repository evolution and execution stability across calibrated
replication environments, recursive execution contexts, and persistence fields.
\end{abstract}

\section{Digital DNA Definition}

Let canonical specification be $P$ and repository state be $R_n$.

A structural fragment $\sigma \subset P$ qualifies as Digital DNA if repeated
execution produces integration-equivalent change-sets:

\[
(P, R_n) \xrightarrow[]{A} R_{n+1}
\]

\[
\mathcal{C}_1 \sim \mathcal{C}_2 \sim \cdots \sim \mathcal{C}_k
\]

Integration equivalence $\sim$ is defined as preservation of:
\begin{itemize}
\item directory topology
\item interface contracts
\item dependency structure
\item integration success without structural conflict
\end{itemize}

\section{Structural Genome}

Let:
\[
P = \{\sigma_1, \sigma_2, \dots, \sigma_m\}
\]

Define genome:
\[
G := \bigcup_i \sigma_i
\]

Genome persistence condition:
\[
G_n = G_0 \quad \forall n
\]

Genome persistence must hold across recursive execution contexts and
execution environments.

\section{Operational Fragment Detection}

A fragment $\sigma$ is invariant if:
\begin{enumerate}
\item Appears across $\ge 3$ executions
\item Survives environment/model perturbation
\item Integrates without structural conflict
\item Preserves topology and interfaces
\item Produces equivalent integration class
\end{enumerate}

Invariant fragments constitute Digital DNA units.

\section{Replication Mapping}

\[
R_{n+1} = F(R_n, G)
\]

Recursive persistence requires mapping invariance across repeated execution.

\section{Drift Metric}

\[
\Delta\Phi_n =
w_t D_{\text{topology}} +
w_i D_{\text{interface}} +
w_d D_{\text{dependency}}
\]

Normalized drift:
\[
\widehat{\Delta\Phi}_n =
\frac{\Delta\Phi_n}{w_t + w_i + w_d}
\]

Total drift:
\[
D = \sum_{n=1}^{k} \widehat{\Delta\Phi}_n
\]

Drift bound:
\[
\widehat{\Delta\Phi}_n \le H_7
\]

Normalized drift ensures calibration invariance across environments.

\section{Stability Functional}

\[
C =
\frac{\text{preserved invariants}}{\text{total invariants}}
\]

\[
S = C - D
\]

Persistence condition:
\[
S > 0
\]

This condition must hold across recursive execution depth and environment
variation.

\section{Genome Stability Field}

Define stability field:
\[
\mathcal{S}(G, E)
\]

where $E$ denotes execution environment class.

Recursive field persistence condition:
\[
\mathcal{S}(G, E_i) > 0 \quad \forall E_i \in \mathcal{E}
\]

and must remain positive under repeated recursive execution:
\[
\mathcal{S}_{k+1}(G, E_i) > 0
\]

\section{Execution Ledger}

Each run emits ledger $L_n$:
\begin{itemize}
\item structural deltas
\item invariant verification
\item drift metrics
\item integration class
\end{itemize}

Heredity condition:
\[
L_n \sim L_0
\]

Ledger continuity anchors replication memory across runs and environments.

\section{Genome Extraction Protocol}

\begin{enumerate}
\item Execute specification across multiple runs
\item Extract invariant fragments
\item Measure drift components
\item Compute stability functional
\item Construct genome $G$
\item Re-execute under perturbation
\item Verify persistence class
\end{enumerate}

\section{Replication Verification}

Fragment qualifies as Digital DNA if:
\[
\widehat{\Delta\Phi}_n \le H_7
\quad\text{and}\quad
\mathcal{C}_n \sim \mathcal{C}_0
\]

across $\ge 3$ runs, environments, and recursive executions.

\section{Digital DNA Principle}

\begin{center}
\fbox{
\parbox{0.85\linewidth}{
\textbf{Postulate.}

Invariant structural fragments within canonical specifications function as
replicable units that reproduce stable integration behavior across repeated
implementations. When invariants persist and drift remains bounded,
repository evolution remains within a stable replication class across
recursive executions, environments, and persistence fields.
}}
\end{center}

\section{Testable Predictions}

\begin{itemize}
\item Stable genomes produce repeatable integrations.
\item Drift correlates with structural ambiguity.
\item Mutation pressure increases with weak invariants.
\item Stability field predicts persistence across environments.
\item Recursive execution preserves genome class under bounded drift.
\end{itemize}

\section*{Closing}

Digital DNA describes structural replication in specification-driven systems.
Invariant fragments act as executable genetic units governing multi-run
repository evolution under bounded drift, recursive execution, and calibrated
stability fields.

\section*{Minimal References}

\begin{thebibliography}{9}

\bibitem{Knuth}
Knuth (1984). Literate Programming.

\bibitem{Hoare}
Hoare (1978). Communicating Sequential Processes.

\end{thebibliography}

\end{document}
