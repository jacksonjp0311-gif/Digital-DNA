% CODEX–DIGITAL DNA SOFTWARE THEORY
% Canonical Structural Replication Genome Engine (DDNA v1.3 · Locked Evolution)
%
% Author: James Paul Jackson
% Date: February 9, 2026
%
% STATUS
% ------
% DIGITAL DNA SOFTWARE THEORY — STRUCTURAL REPLICATION ENGINE FORMALIZATION.
%
% This document formalizes Digital DNA as a software theory describing how
% invariant structural fragments within canonical specifications reproduce
% stable integration behavior across repeated AI-assisted repository
% implementations under bounded drift and recursive execution.
%
% • Canonical structure encodes replicable structural fragments.
% • Structural invariants function as executable genome units.
% • Recursive execution expresses genome into repository state.
% • Translation drift acts as mutation pressure.
% • Stability bounds govern replication class persistence.
% • Execution ledgers anchor structural heredity across runs.
%
% Digital DNA Software Theory denotes a measurable system in which invariant
% structural fragments reproduce stable integration behavior across recursive
% execution environments within bounded drift.
%
% No biological or metaphysical interpretation is made. This is a software
% theory for structural replication verification in specification-driven
% implementation systems.
%
% v1.3 establishes:
% • Structural genome extraction operator
% • Multi-run replication class detection
% • Drift-bounded stability functional
% • Cross-environment persistence verification
% • Ledger-anchored heredity continuity
% • Stability-field recursion guarantee
% • Fragment equivalence operator
% • Drift inevitability null theorem
% • Calibration protocol note
%
% EXTRACTION
% ----------
% Artifact Class:
% Canonical specifications; structural schemas; repository state graphs;
% dependency surfaces; runtime artifacts; execution ledgers.
%
% ★ Codex Extraction Framework:
%
% (1) Identify replication boundary
% (2) Detect invariant fragment σ
% (3) Construct structural genome G
% (4) Measure drift ΔΦ across runs
% (5) Compute stability functional S
% (6) Execute recursive replication
% (7) Verify replication class persistence
%
% WHAT THIS IS
% ------------
% • A structural replication software theory.
% • A deterministic multi-run stability model.
% • A measurable repository heredity system.
% • A drift-bounded genome persistence framework.
%
% WHAT THIS IS NOT
% ----------------
% • Not biology.
% • Not metaphysical.
% • Not model-dependent.
% • Not a code-generation system.
%
% NULL PREDICTIONS
% ----------------
% Unstructured specifications fail to replicate integration behavior.
% Weak invariants increase mutation drift across runs.
% Stable genomes produce repeatable integration classes.
% Persistent behavior emerges only under invariant-bounded execution.
%
% ─────────────────────────────────────────────────────────────────────────────

\documentclass[12pt]{article}
\usepackage{amsmath,amssymb}
\usepackage{geometry}
\usepackage[pdftex]{hyperref}
\geometry{margin=1in}

\newcommand{\keywords}{
digital DNA; structural replication; repository evolution;
drift metrics; invariant detection; execution ledger;
integration stability; genome persistence.
}
\newcommand{\msc}{
Primary 68N01; Secondary 68T01, 68Q99.
}

\title{\textbf{Digital DNA Software Theory: Structural Genome Persistence in Recursive Systems}}
\author{\textbf{James Paul Jackson}}
\date{February 9, 2026}

\begin{document}
\maketitle

\vspace{-0.6em}
\noindent\textbf{Keywords:} \keywords
\noindent\textbf{MSC (2020):} \msc

\begin{abstract}
We formalize Digital DNA as a software theory in which invariant specification
fragments function as persistent structural genome units that reproduce stable
repository integration behavior across repeated AI-assisted implementations.
When structural invariants are preserved and translation drift remains bounded,
successive implementations remain within a stable replication class. The theory
emits ledger-anchored evidence of genome persistence, drift metrics, and
integration class continuity across recursive execution environments and
multi-run replication fields.
\end{abstract}

\section{Digital DNA Definition}

Let canonical specification be $P$ and repository state be $R_n$.

Execution operator:
\[
(P, R_n) \xrightarrow[]{A} R_{n+1}
\]

A structural fragment $\sigma \subset P$ qualifies as Digital DNA if repeated
execution produces integration-equivalent change sets:
\[
\mathcal{C}_1 \sim \mathcal{C}_2 \sim \cdots \sim \mathcal{C}_k
\]

Integration equivalence $\sim$ preserves:
\begin{itemize}
\item directory topology
\item interface contracts
\item dependency structure
\item artifact emission class
\item integration success
\end{itemize}

\section{Fragment Equivalence Operator}

Define structural equivalence:
\[
\sigma_i \approx \sigma_j
\]

if structural distance:
\[
d(\sigma_i,\sigma_j) < \epsilon
\]

Distance may be computed via:
\begin{itemize}
\item tree-edit distance
\item AST signature comparison
\item dependency graph overlap
\item artifact family match
\end{itemize}

Genome intersection:
\[
G_n \cap G_0 := \{\sigma \in G_n : \exists \sigma_0 \in G_0 \text{ s.t. } \sigma \approx \sigma_0\}
\]

\section{Structural Genome}

Let:
\[
P = \{\sigma_1, \sigma_2, \dots, \sigma_m\}
\]

Define genome:
\[
G := \bigcup_i \sigma_i
\]

Genome persistence condition:
\[
G_n \sim G_0 \quad \forall n
\]

\section{Drift Metric}

\[
\Delta\Phi_n =
w_t D_{\text{topology}} +
w_i D_{\text{interface}} +
w_d D_{\text{dependency}} +
w_a D_{\text{artifact}} +
w_r D_{\text{runtime}}
\]

Normalized drift:
\[
\widehat{\Delta\Phi}_n =
\frac{\Delta\Phi_n}{w_t + w_i + w_d + w_a + w_r}
\]

\subsection*{Calibration}
$H_7$ is empirically calibrated from baseline multi-run replication experiments
within a given ecosystem and may vary by domain.

Drift bound:
\[
\widehat{\Delta\Phi}_n \le H_7
\]

\section{Stability Functional}

Invariant retention:
\[
C =
\frac{|G_n \cap G_0|}{|G_0|}
\]

Cumulative drift:
\[
D = \sum_{n=1}^{k} \widehat{\Delta\Phi}_n
\]

Stability functional:
\[
S = C - D
\]

Persistence condition:
\[
S > 0
\]

Recursive stability theorem:
\[
S_{k+1} > 0 \quad \forall k
\]

implies indefinite replication-class persistence under bounded drift.

\section{Drift Inevitability Null Theorem}

If cumulative drift exceeds unity:
\[
D > 1
\]

then:
\[
\exists m \le k \text{ such that } C_m < C_0
\]

i.e., replication class degradation is inevitable.

\section{Integration Class System}

\[
\mathcal{C} \in \{C_0, C_1, C_2, C_3, C_4, C_5\}
\]

\begin{itemize}
\item $C_0$: incomplete
\item $C_1$: builds
\item $C_2$: runs
\item $C_3$: emits artifacts
\item $C_4$: stable
\item $C_5$: replicates across environments
\end{itemize}

\section{Execution Ledger}

Each run emits ledger $L_n$ containing:
\begin{itemize}
\item structural deltas
\item invariant verification
\item drift metrics
\item integration class
\item environment signature
\end{itemize}

Ledger continuity anchors replication memory across runs.

\section{Replication Verification}

Fragment qualifies as Digital DNA if:
\[
\widehat{\Delta\Phi}_n \le H_7
\quad\text{and}\quad
\mathcal{C}_n \sim \mathcal{C}_0
\]

across $\ge 3$ runs and $\ge 2$ environments.

\section{Digital DNA Software Theory Principle}

\begin{center}
\fbox{
\parbox{0.85\linewidth}{
\textbf{Postulate.}

Invariant structural fragments within canonical specifications function as
replicable genome units that reproduce stable integration behavior across
repeated implementations. When invariants persist and drift remains bounded,
repository evolution remains within a stable replication class across
recursive executions and environments.
}}
\end{center}

\section{Testable Predictions}

\begin{itemize}
\item Stable genomes produce repeatable integrations.
\item Drift correlates with structural ambiguity.
\item Mutation pressure increases with weak invariants.
\item Stability field predicts persistence across environments.
\item Recursive execution preserves genome class under bounded drift.
\end{itemize}

\section*{Closing}

Digital DNA Software Theory formalizes structural replication verification in
specification-driven systems. Invariant fragments act as executable genome
units governing multi-run repository evolution under bounded drift, recursive
execution, and ledger-anchored stability fields.

\section*{Minimal References}

\begin{thebibliography}{9}

\bibitem{Knuth}
Knuth (1984). Literate Programming.

\bibitem{Hoare}
Hoare (1978). Communicating Sequential Processes.

\end{thebibliography}

\end{document}
